%
% 6.006 problem set 0 solutions template
%
\documentclass[12pt,twoside]{article}

\input{macros-sp20}
\newcommand{\theproblemsetnum}{0}

\title{6.006 Problem Set 0}

\begin{document}

% \handout{Problem Set \theproblemsetnum}

\setlength{\parindent}{0pt}
\medskip\hrulefill\medskip

{\bf Name:} LessTanker

\medskip\hrulefill

%%%%%%%%%%%%%%%%%%%%%%%%%%%%%%%%%%%%%%%%%%%%%%%%%%%%%
% See below for common and useful latex constructs. %
%%%%%%%%%%%%%%%%%%%%%%%%%%%%%%%%%%%%%%%%%%%%%%%%%%%%%

% Some useful commands:
% $f(x) = \Theta(x)$
% $T(x, y) \leq \log(x) + 2^y + \binom{2n}{n}$
% \ttt{code\_function}


% You can create unnumbered lists as follows:
% \begin{itemize}
%     \item First item in a list
%         \begin{itemize}
%             \item First item in a list
%                 \begin{itemize}
%                     \item First item in a list
%                     \item Second item in a list
%                 \end{itemize}
%             \item Second item in a list
%         \end{itemize}
%     \item Second item in a list
% \end{itemize}

% You can create numbered lists as follows:
% \begin{enumerate}
%     \item First item in a list
%     \item Second item in a list
%     \item Third item in a list
% \end{enumerate}

% You can write aligned equations as follows:
% \begin{align}
%     \begin{split}
%         (x+y)^3 &= (x+y)^2(x+y) \\
%                 &= (x^2+2xy+y^2)(x+y) \\
%                 &= (x^3+2x^2y+xy^2) + (x^2y+2xy^2+y^3) \\
%                 &= x^3+3x^2y+3xy^2+y^3
%     \end{split}
% \end{align}

% You can create grids/matrices as follows:
% \begin{align}
%     A =
%     \begin{bmatrix}
%         A_{11} & A_{21} \\
%         A_{21} & A_{22}
%     \end{bmatrix}
% \end{align}

\begin{problems}

\problem  % Problem 1

\begin{problemparts}
\problempart % Problem 1a
\(A=\{1,6,12,13,9\},B=\{3,6,12,15\}\)
As a result, $A\cap B=\{6,12\}$
\problempart % Problem 1b
Also easy to find that $A\cup B=\{3,6,9,12,13,15\}$,so answer is 7.
\problempart % Problem 1c
\(|A-B|=\{1,9,13\}\) Answer is 3.
\end{problemparts}

\problem  % Problem 2

\begin{problemparts}
\problempart % Problem 2a
\(E[X]=\frac{3}{2}\)
\problempart % Problem 2b
\(E[Y]=12.25\)
\problempart % Problem 2c
\(E[X+Y]=13.75\)
\end{problemparts}

\problem  % Problem 3

\begin{problemparts}
\problempart % Problem 3a
True
\problempart % Problem 3b
False
\problempart % Problem 3c
False
\end{problemparts}

\problem  % Problem 4
Assume that for k, the cite is true.Then for the k+1 case:
\[\sum_{i=1}^{k+1}i^3=\sum_{i=1}^{k}i^3+(k+1)^3=[\frac{k(k+1)}{2}]^2+(k+1)^3=[\frac{(k+1)(k+2)}{2}]^2\]
Therefore we prove that cite.
\newpage
\problem  % Problem 5
Assume that we have a graph with $e_0$ edges and $v_0$ vertices while $e_0=v_0-1$.
Then we add a vertice to that graph, we can find that we must add two more edges to make
this new graph acyclic. As a result, the original graph must be acyclic.
\vfill
\problem  % Problem 6
Submit your implementation to {\small\url{alg.mit.edu}}.

\begin{lstlisting}
def count_long_subarray(A):
    '''
    Input:  A     | Python Tuple of positive integers
    Output: count | number of longest increasing subarrays of A
    '''
    count = 1
    ##################
    # YOUR CODE HERE #
    ##################
    max_len = 1
    cur_count = 1
    for i in range(1,len(A)-1):
        if A[i] > A[i-1] :
            cur_count += 1
        else:
            cur_count = 1

        if cur_count > max_len :
            max_len = cur_count
            count = 1
        elif cur_count == max_len:
            count += 1
    return count
\end{lstlisting}

\end{problems}

\end{document}
