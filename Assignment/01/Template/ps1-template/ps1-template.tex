%
% 6.006 problem set 1 solutions template
%
\documentclass[12pt,twoside]{article}

\input{macros-sp20}
\newcommand{\theproblemsetnum}{1}

\title{6.006 Problem Set 1}

\begin{document}

% \handout{Problem Set \theproblemsetnum}

\setlength{\parindent}{0pt}
\medskip\hrulefill\medskip

{\bf Name:} LessTanker

\medskip

{\bf Collaborators:} Name1, Name2

\medskip\hrulefill

%%%%%%%%%%%%%%%%%%%%%%%%%%%%%%%%%%%%%%%%%%%%%%%%%%%%%
% See below for common and useful latex constructs. %
%%%%%%%%%%%%%%%%%%%%%%%%%%%%%%%%%%%%%%%%%%%%%%%%%%%%%

% Some useful commands:
%$f(x) = \Theta(x)$
%$T(x, y) \leq \log(x) + 2^y + \binom{2n}{n}$
% {\tt code\_function}


% You can create unnumbered lists as follows:
%\begin{itemize}
%    \item First item in a list
%        \begin{itemize}
%            \item First item in a list
%                \begin{itemize}
%                    \item First item in a list
%                    \item Second item in a list
%                \end{itemize}
%            \item Second item in a list
%        \end{itemize}
%    \item Second item in a list
%\end{itemize}

% You can create numbered lists as follows:
%\begin{enumerate}
%    \item First item in a list
%    \item Second item in a list
%    \item Third item in a list
%\end{enumerate}

% You can write aligned equations as follows:
%\begin{align}
%    \begin{split}
%        (x+y)^3 &= (x+y)^2(x+y) \\
%                &= (x^2+2xy+y^2)(x+y) \\
%                &= (x^3+2x^2y+xy^2) + (x^2y+2xy^2+y^3) \\
%                &= x^3+3x^2y+3xy^2+y^3
%    \end{split}
%\end{align}

% You can create grids/matrices as follows:
%\begin{align}
%    A =
%    \begin{bmatrix}
%        A_{11} & A_{21} \\
%        A_{21} & A_{22}
%    \end{bmatrix}
%\end{align}

% You can include images and PDFs as follows:
% \includegraphics[width=0.5\textwidth]{img.jpg}

\begin{problems}

\problem  % Problem 1

\begin{problemparts}
\problempart % Problem 1a
\((f_5,f_3,f_4,f_2,f_1)\)
\problempart % Problem 1b
\((\{f_1,f_2\},f_5,f_4,f_3)\)
\problempart % Problem 1c
\((\{f_2,f_5\},f_4,f_3,f_1)\)
\problempart % Problem 1d
\((f_5,f_3,f_4,f_2,f_1)\)
\end{problemparts}

\newpage
\problem  % Problem 2

\begin{problemparts}
\problempart % Problem 2a
In a for loop from j=0 to k-1, call insert\_at(i+k-j-1,delete\_at(i)).

My idea is in a loop, each time we move one element from front to the end,
then the first place`s  index will always be i, and the last index to reverse is correspond
to j, and the index to insert in is i+k-j-1.
\problempart % Problem 2b
In a for loop from it=0 to k-1,call insert\_at(j-it-1,delete\_at(i+k-it-1)).

Similiar to the above one, my solution is to delete the original list from the end,
then move the element to the front of index j.
\end{problemparts}

\newpage
\problem  % Problem 3
My solution is a linked list with index inside the value.

For initialization, obviously a linked list takes O(x) time to do.
place\_mark(i,m) can even go to O(1) time, while read\_page(i),shift\_mark(m,d) and
move\_page(m) is just basic linked list operation which all takes O(1) time.
\newpage
\problem  % Problem 4

\begin{problemparts}
\problempart % Problem 4a
insert\_first(x) needs to let head.prev = x , x.next = head, x.prev = null.

insert\_last(x) needs to let tail.next = x , x.prev = tail, x.next=null.

delete\_first() needs to set head.next.prev = null.

delete\_last() needs to set last.prev.next = null.
\problempart % Problem 4b
let a = x1.next, b=x2.prev.

Then set a.prev = null, b.next=null.

Next,let x1.next = x2, x2.prev = x1

Finally, return a as the head of the new doubly linked list, b as the tail.
\problempart % Problem 4c
Remember x.next as y,then set x.next = L2.head, L2.head.prev = x,L2.tail.next = y,
y.prev=L2.tail.
\problempart Submit your implementation to {\small\url{alg.mit.edu}}.
\end{problemparts}

\end{problems}

\end{document}
