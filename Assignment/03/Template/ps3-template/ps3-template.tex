%
% 6.006 problem set 3 solutions template
%
\documentclass[12pt,twoside]{article}

\input{macros-sp20}
\newcommand{\theproblemsetnum}{3}

\title{6.006 Problem Set \theproblemsetnum}

\begin{document}

%\handout{Problem Set \theproblemsetnum}

\setlength{\parindent}{0pt}
\medskip\hrulefill\medskip

{\bf Name:} LessTanker

\medskip

{\bf Collaborators:} ChatGPT, Name2

\medskip\hrulefill

%%%%%%%%%%%%%%%%%%%%%%%%%%%%%%%%%%%%%%%%%%%%%%%%%%%%%
% See below for common and useful latex constructs. %
%%%%%%%%%%%%%%%%%%%%%%%%%%%%%%%%%%%%%%%%%%%%%%%%%%%%%

% Some useful commands:
%$f(x) = \Theta(x)$
%$T(x, y) \leq \log(x) + 2^y + \binom{2n}{n}$
% {\tt code\_function}


% You can create unnumbered lists as follows:
%\begin{itemize}
%    \item First item in a list
%        \begin{itemize}
%            \item First item in a list
%                \begin{itemize}
%                    \item First item in a list
%                    \item Second item in a list
%                \end{itemize}
%            \item Second item in a list
%        \end{itemize}
%    \item Second item in a list
%\end{itemize}

% You can create numbered lists as follows:
%\begin{enumerate}
%    \item First item in a list
%    \item Second item in a list
%    \item Third item in a list
%\end{enumerate}

% You can write aligned equations as follows:
%\begin{align}
%    \begin{split}
%        (x+y)^3 &= (x+y)^2(x+y) \\
%                &= (x^2+2xy+y^2)(x+y) \\
%                &= (x^3+2x^2y+xy^2) + (x^2y+2xy^2+y^3) \\
%                &= x^3+3x^2y+3xy^2+y^3
%    \end{split}
%\end{align}

% You can create grids/matrices as follows:
%\begin{align}
%    A =
%    \begin{bmatrix}
%        A_{11} & A_{21} \\
%        A_{21} & A_{22}
%    \end{bmatrix}
%\end{align}

% You can include images and PDFs as follows:
% \includegraphics[width=0.5\textwidth]{img.jpg}

\begin{problems}

\problem  % Problem 1

\begin{problemparts}
\problempart % Problem 1a
The picture is as follows:\\
\includegraphics[width=0.5\textwidth]{1.jpg}
\problempart % Problem 1b
13: Calculate from the smallest prime p with $p>10$ ,11 is not ok, then 13 does.
\end{problemparts}

% \newpage

\problem  % Problem 2

\begin{problemparts}
\problempart % Problem 2a
\[k_1 = k_2 + t\cdot n(t\in \mathbb{N})\]
\problempart % Problem 2b
Choose $k_1$ and $k_2$ to be adjacent so that 
\[|\frac{k_1n}{u}| = |\frac{k_2n}{u}|\]
\problempart % Problem 2c
We can`t  guarantee that they will be roommates because in a universal hash family,
the probability of two keys collide is at most $\frac{1}{n}$.
\end{problemparts}

% \newpage

\problem  % Problem 3

\begin{problemparts}
\problempart % Problem 3a
Using radix sort to make this $O(n)$
\problempart % Problem 3b
radix sort again?
\problempart % Problem 3c
IF n is a constant, then use radix sort by multipling $n^3$
\problempart % Problem 3d
Merge sort this time because we need to compare.
\end{problemparts}

\newpage

\problem  % Problem 4

\begin{problemparts}
\problempart % Problem 4a
hash
\problempart % Problem 4b
Radix sort first, then use two pointers, the first one marks the head, the other one marks the tail.
If the total of two pointers is bigger than $r$, then move tail pointer one step back.
Otherwise, move head pointer one step forward till find the closest one.
\end{problemparts}

\newpage

\problem  % Problem 5

\begin{problemparts}
\problempart % Problem 5a

\problempart % Problem 5b
\problempart Submit your implementation to {\small\url{alg.mit.edu}}.
\end{problemparts}

\end{problems}

\end{document}
